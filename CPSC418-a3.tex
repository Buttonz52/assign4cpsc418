\documentclass{assignment}

\coursetitle{Introduction to Cryptography}
\courselabel{CPSC 418}
\exercisesheet{Home Work \#[4]}{}
\student{Brendan Petras - 10137098}
\semester{Fall 2016}
%\usepackage[pdftex]{graphicx}
%\usepackage{subfigure}
\usepackage{amsmath}
\usepackage{cancel}
\usepackage{amsmath}

\begin{document}

\begin{center}
\renewcommand{\arraystretch}{2}
\begin{tabular}{|c|c|c|} \hline
Problem & Marks \\ \hline \hline
1 & \\ \hline
2 & \\ \hline
3 & \\ \hline
4 & \\ \hline
5 & \\ \hline
6 & \\ \hline
7 & \\ \hline \hline
Total & \\ \hline
\end{tabular}
\end{center}

\bigskip

\clearpage

\begin{problemlist}
\pbitem
\begin{problem}
\begin{enumerate}

\item
$\Phi(n) = (p-1)(q-1) = pq - p - q + 1$ \newline
$\Phi(n) = n - p - \frac{n}{p} + 1$ \newline
$\Phi(n)p = np - p^2 - n + p$ \newline
$0 = np - p^2 - n + p -\Phi(n)p$ - rearrange \newline
$0 = p^2 + \Phi(n)p - np - p + n$ - group like terms \newline
$0 = p^2 + (\Phi(n) - n - 1)p + n$ \newline
We can find the roots of this quadratic which will return p and q. \newline
With, $a = 1, b = \Phi(n) - n - 1, c = n$ \newline
$x=\frac{-b\pm\sqrt{b^2-4ac}}{2a}$, x will return two solutions, namely $p$ and $q$, assuming  $\sqrt{b^2-4ac} \neq 0.$\newline

\item
Start with $M = M^1$ \newline
Notice that $gcd(e_1,e_2) = 1$ means that $e_1x + e_2y = 1$ \newline
Notice that $M^1 = M^{e_1x + e_2y}$ \newline
$ = M^{e_1x} * M^{e_2y}$ \newline
$ = C_1^{x} * C_2^{y}$ - since $M^{e_1} = C_1$ and $M^{e_2} = C_2$  \newline
Since $C_1,C_2,x,y$ are known we can find $M$.\newline

\item
Given $(e,n_1),(e,n_2),...,(e,n_i)$ and $gcd(n_i,n_j) = 1$, with $i \neq j$\newline
In general, $C_i \equiv M^e$ (mod $n_i$), for all $i$. Assume $k \geq e$ where $k$ is number of participants. \newline
By The Chinese Remainder Theorem, we can efficiently compute $C$, \newline
with $C \equiv C_i$ (mod $n_i$) for $1 \geq i \geq k$. So $C \equiv M^e$ (mod $n$) with ($n = \prod_{1}^{i} n_i$). \newline
Which is to say $M^e$ is not modded by any $n_i$ \newline
Thus finding $C_i = (M^e)^\frac{1}{e}$ is a eth root calculation \newline

\item
Given $e = 3$ Bob wants to send $M, M+r,M+s$ o Alice \newline
$M$ - unknown, $r,s,C,C_r,C_s$ - known \newline
$C  \equiv M^3$ (mod $n$) \newline
$C_r \equiv (M + r)^3$  (mod $n$) \newline
$C_s \equiv (M + s)^3$  (mod $n$) \newline

$C \equiv M^3$  (mod $n$) \newline
$C_r \equiv (M + r)^3 \equiv M^3 + 3M^2r + 3Mr^2 + r^3$  (mod $n$) \newline
$C_s \equiv (M + s)^3 \equiv M^3 + 3M^2s + 3Ms^2 + s^3$  (mod $n$) \newline
$C_r \equiv C_r - C \equiv 3M^2r + 3Mr^2 + r^3$  (mod $n$) \newline
$C_s \equiv C_s - C \equiv 3M^2s + 3Ms^2 + s^3$  (mod $n$) \newline
$\equiv 3M^2r + 3Mr^2$  (mod $n$) - remove $r^3$ since it is a known constant \newline
$\equiv 3M^2s + 3Ms^2$  (mod $n$) - remove $s^3$ since it is a known constant \newline
$\equiv 3Mr(M + r)$  (mod $n$) - factor $Mr$\newline
$\equiv 3Ms(M + s)$  (mod $n$) - factor $Ms$ \newline
$\equiv 3Mr\sqrt[3]{C_r}$  (mod $n$) - sub $C_r$ \newline
$\equiv 3Ms\sqrt[3]{C_s}$  (mod $n$) - sub $C_s$ \newline
$M \equiv 1(3r\sqrt[3]{C_r})^{-1}$  (mod $n$) - modular inverse $C_r$ \newline
$M \equiv 1(3s\sqrt[3]{C_s})^{-1}$  (mod $n$) - modular inverse $C_s$ \newline

\end{enumerate}
\clearpage
\end{problem}

\begin{problem}
\pbitem

\begin{enumerate}

Prove CRT decrypts like normal, prove $m^{\prime} = m$ \newline
$ m \equiv pxM_q + qyM_p$ (mod $n$) \newline
$ m \equiv qyM_p$ (mod $p$) \newline
$ m \equiv M_p$ (mod $p$) - since $gcd(px + qy) = 1 -> gy \equiv 1$ (mod $p$)  \newline
$ m \equiv pxM_q$  (mod $q$) \newline
$ m \equiv M_q$ (mod $p$) - since $gcd(px + qy) = 1 -> px \equiv 1$ (mod $q$)  \newline

$m^{\prime} \equiv C^d$ (mod $n$) \newline
$m^{\prime} \equiv C^{d_p + \cancel{(p - 1)L}}$ (mod $p$) - by Fermat's LT \newline
$m^{\prime} \equiv M_p$ (mod $p$) \newline
$m^{\prime} \equiv C^{d_q + \cancel{(q - 1)T}}$ (mod $q$) - by Fermat's LT \newline
$m^{\prime} \equiv M_q$ (mod $q$)\newline

So, $gcd(p,q) = 1$ \newline
$px + qy = 1$ \newline
$M(px + qy) = M$ \newline
$Mpx + Mqy = M$ \newline
$M_qpx + M_pqy \equiv M$ (mod $n$) \newline

 

\end{enumerate}


\clearpage
\end{problem}

\begin{problem}
\pbitem

Prove that the cryptosystem is NOT IND-CCA

A cryptosystem is not IND-CCA secure if some active attacker, with some decryption oracle can, in polynomial time, select two plaintexts $M_1, M_2$ and correctly distinguish between the encryptions of $M_1$ and $M_2$ with probability significantly greater than 1/2.

Start with $M_1,M_2$ with $M_1 \neq 0^n$ \newline
Apply the attack $C^{\prime} =  (s||t \oplus M_1)$, we can examine $C^{\prime}$ since we have a decryption oracle. \newline

$C^{\prime} =  (s||t \oplus M_1)$ \newline
$C^{\prime} =  (s||H(r) \oplus M_1\oplus M_i)$ \newline
if $i=1$ then $C^{\prime} =  (s||H(r))$ \newline

Apply the decryption function $D$ \newline

$D(C^{\prime}) = H(s^d) \oplus H(r)$ (mod $n$)\newline
$D(C^{\prime}) = H(r^{ed}) \oplus H(r)$ (mod $n$) \newline
$D(C^{\prime}) = H(r) \oplus H(r)$ (mod $n$) \newline
$D(C^{\prime}) = ^m$ \newline

Then we know with 100\% probability which plaintext the ciphertext belongs to. \newline
Thus it is not IND-CCA secure.

\clearpage
\end{problem}


\begin{problem}
\pbitem
\begin{enumerate}
\item

\begin{enumerate}

\item

Given $(r,s_1),(r,s_2),M_1,M_2,Hash function$ $H$. Find $k$. \newline

$ks_1 \equiv H(M_1,r) - xr$ (mod $p-1$) \newline
$ks_2 \equiv H(M_1,r) - xr$ (mod $p-1$) \newline
$ks_1 - ks_2 + H(M_1,r) - xr - H(M_2,r) + xr + (p - 1)L - (p - 1)T = 0$ converting from congruence to equality. With L,T real numbers. \newline
$ks_1 - ks_2 + H(M_1,r) - xr - H(M_2,r) + xr + (p - 1)L - (p - 1)T = 0$\newline
$k(s_1 - s_2) + H(M_1,r) - H(M_2,r) + \cancel{xr} - \cancel{xr} + (p - 1)L - (p - 1)T = 0$\newline
$k(s_1 - s_2) \equiv H(M_2,r) - H(M_1,r)$ (mod $p-1$) - move back to congruence \newline
$k \equiv [H(M_2,r) - H(M_1,r)] (s_1 - s_2)^{-1}$ (mod $p-1$) - modular inverse since $gcd(s_1 - s_2, p-1) = 1$\newline

\item

We now know k. \newline

$ks_1 \equiv H(M_1,r) - xr$ (mod $p-1$)\newline
$xr \equiv H(M_1,r) -ks_1$ (mod $p-1$)\newline
$x \equiv r^{-1}[H(M_1,r) -ks_1]$ (mod $p-1$) - since $gcd(r,p-1) = 1$\newline

\end{enumerate}

\item

\begin{enumerate}

\item
Prove $y^rr^s \equiv g^m$ (mod $p-1$) \newline
$y^rr^s \equiv g^m$ (mod $p-1$)  \newline
$y^rg^{su}y^{vs} \equiv g^{su}$ (mod $p-1$)  \newline
$y^ry^{vs}g^{su} \equiv g^{su}$ (mod $p-1$) - now lets try getting $y^ry^{vs}$ to equal 1 \newline
$y^{r + vs}g^{su} \equiv g^{su}$ (mod $p-1$) \newline
$y^{r + -rv^{*}v}g^{su} \equiv g^{su}$ (mod $p-1$) \newline
$y^{(r + -r)1}g^{su} \equiv g^{su}$ (mod $p-1$) - since $vv^* \equiv 1$ (mod $p-1$) \newline
$g^{su} \equiv g^{su}$ (mod $p-1$)\newline

\item
When we return the hash function H back into ElGamal, we no longer can find a $m$ in $g^{H(M,r)}$ such that $m = H(M,r)$. This is the deffinition of pre-image resistance.

\end{enumerate}

\clearpage

\item

\begin{enumerate}

\item
Prove $R \equiv ru$ (mod $p-1$) \newline
$ R = rup -r(p-1) + p(p-1)L$ - for some real $L$  \newline
$ R \equiv rup + p^2 - p$ (mod $p-1$) \newline
$ R \equiv p(ru + p - 1)$ (mod $p-1$) \newline
$ R \equiv rup + \cancel{p(p - 1)}$ (mod $p-1$) \newline
$ R \equiv ru(p - 1 + 1)$ (mod $p-1$) \newline
$ R \equiv \cancel{ru(p - 1)} + ru)$ (mod $p-1$) \newline
$ R \equiv ru$ (mod $p-1$) \newline

\item
Prove $R^S \equiv r^{su}$ (mod $p$) \newline
$R^S \equiv r^{su}$ (mod $p$) \newline
$R^{su} \equiv [rup - r(p-1)]^{su}$ (mod $p$) \newline
$R^{su} \equiv (\cancel{rup^{su}} -\cancel{rp^{su}} + r^{su})$ (mod $p$) \newline
$R^{su} \equiv  r^{su}$ (mod $p$) \newline
$R^{S} \equiv  r^{su}$ (mod $p$) \newline


\item
Prove $y^RR^S \equiv g^{H(M^{\prime})}$ (mod $p$) - This shows (R,S) is a valid signature to message $M^{\prime}$ \newline
$y^RR^S \equiv g^{H(M^{\prime})}$ (mod $p$) \newline
$y^{ru}r^{su} \equiv g^{H(M^{\prime})}$ (mod $p$) \newline
$y^{ru}g^{ksu} \equiv g^{H(M^{\prime})}$ (mod $p$) \newline
$g^{xru}g^{ksu} \equiv g^{H(M^{\prime})}$ (mod $p$) \newline
$g^{xru}g^{[H(M) - xr]u} \equiv g^{H(M^{\prime})}$ (mod $p$) \newline
$g^{xru}g^{H(M)u} g^{-xru} \equiv g^{H(M^{\prime})}$ (mod $p$) \newline
$\cancel{g^{xru}}g^{H(M)u} \cancel{g^{-xru}} \equiv g^{H(M^{\prime})}$ (mod $p$) \newline
$g^{H(M)u} \equiv g^{H(M^{\prime})}$ (mod $p$) \newline
$g^{H(M^{\prime})} \equiv g^{H(M^{\prime})}$ (mod $p$) - via step 3 and the EEA. \newline





\end{enumerate}
\end{enumerate}
\end{problem}

\end{problemlist}
\end{document}
